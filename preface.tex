

\chapter*{About this template}
\label{chap:about}

This document, along with the source code, is meant to be a self-explanatory template for a bachelor's or master's thesis in computer science\footnote{By minor modifications it could be used in any field of research}. It is implemented in \LaTeX, the most popular, advanced, and comprehensive documentation system in mathematics and natural sciences. However, the report as such could be implemented with other tools, as well.

The suggested structure is based on the widely used IMRAD model (Introduction, Methods, Results, And Discussion)\footnote{\url{en.wikipedia.org/wiki/IMRAD}}. The student may of course deviate from the structure and recommended content for each chapter. In particular, the chapters describing the main bulk of work done in the research project (Chapters \ref{chap:design}, \ref{chap:implementation}, and \ref{chap:evaluation}), should be customized to fit the specific topic of your project, both regarding chapter titles and content. You may also, of course, consider merging some of the chapters, and/or add more chapters.

The report is based on the author's personal experiences as a research scientist and lecturer during the last 25 years\footnote{\url{www.ia.hiof.no/~gunnarmi}}, various online resources, 
and the ``The Mayfield Handbook of Technical and Scientific Writing"\cite{perelman97mht}\footnote{\url{www.mhhe.com/mayfieldpub/tsw/home.htm}}.

For technical details, see Chapter \ref{chap:how-to}.

Finally: Comments, bug reports, and suggestions are highly welcome\footnote{\url{gunnar.misund@hiof.no}}!

\vspace{20mm}

Gunnar Misund

Halden, \today



