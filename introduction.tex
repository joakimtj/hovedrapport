
\chapter{Introduction}
\label{chap:intro}

Chess-playing robots represent a fascinating intersection of robotics, computer spacial recognition, and artificial intelligence. These systems combine mechanical precision with computational intelligence to create machines capable of engaging in one of humanity's oldest strategic games. While chess computers have famously defeated grandmasters since the late 1990s, the physical manipulation of chess pieces by robotic systems presents unique engineering challenges beyond pure computational play.

\section{Background and motivation}
\label{sec:background-motivation}

The development of chess-playing robots has evolved along several technological paths. Previous research has explored various approaches for piece detection and movement systems. For instance, solutions demonstrate a vision-based detection system using overhead cameras to identify piece positions and a robotic arm for movement, while other papers also shed light on systems utilizing magnetic sensors with various movement mechanism for piece manipulation.
Our project aims to combine the strengths of these approaches while introducing novel elements. By implementing hall effect sensors for piece detection and integrating an articulated robotic arm for movement, we create a system that offers reliable position tracking without the lighting dependencies of vision systems, while maintaining the flexibility of a multi-axis robotic manipulator for piece movement.

TODO: Further elaboration needed on the specific advantages of hall effect sensors over camera-based solutions and why this approach is interesting

\subsection{Research question/Problem statement/Objectives}
\label{sec:research-question}

The primary objective of this project is to develop an integrated chess-playing robot system.
\begin{description}
  \item[Objective 1] To provide a seamless interface between the detection system, chess engine, and robotic control mechanisms
        \begin{description}
          \item[Objective 1.1] To accurately detect the position of all chess pieces on the board using hall effect sensors
          \item[Objective 1.2] To process the board state and determining optimal moves using the Stockfish chess engine
          \item[Objective 1.3] To precisely manipulate chess pieces using the supplied robotic arm
        \end{description}
\end{description}

TODO: Further elaboration needed on specific performance metrics and success criteria

\subsection{Method}
\label{sec:method}

Our approach to developing a chess-playing robot system involves three primary components working in concert:
\begin{itemize}
  \item A chessboard equipped with hall effect sensors to detect piece positions
  \item A computational interface running Stockfish to calculate optimal chess moves
  \item A robotic arm (provided by the school) for physical piece manipulation
\end{itemize}

The development process includes designing the sensor array for the chessboard, creating software interfaces between all components, calibrating the robotic arm for precise movement, and extensive testing of the integrated system.

TODO:  Further elaboration needed on specific technical implementation details and development methodology

\subsection{Deliverables}
\label{sec:deliverables}

The project will result in the following deliverables:
\begin{itemize}
  \item A functional chess-playing robot system integrating all three components
  \item Software for the detection system and computational interface
  \item Technical documentation of the system architecture and integration
  \item Performance analysis and evaluation
  \item This comprehensive report documenting our approach, implementation, and findings
\end{itemize}

Beyond academic objectives, this chess-playing robot has several practical applications. As an educational tool, it can demonstrate chess strategies to beginners while providing a tangible interface for learning. In human-robot interaction research, the system offers a structured environment to study collaborative behavior and user experience with robotic systems. Additionally, the project serves as a valuable testbed for sensor fusion techniques and precision movement algorithms applicable to industrial automation and assistive robotics. These broader applications enhance the project's impact beyond the immediate technical accomplishments of the chess-playing capabilities.

\section{Report Outline}

Chapter 2 examines the current state of chess-playing robots, providing detailed comparisons between detection technologies with focus on hall effect sensors versus vision-based systems. It presents system requirements derived from this analysis and establishes performance criteria. The third chapter details the system architecture, including sensor array design, software interfaces, and integration with the robotic arm. This chapter explains design decisions and component selection rationale to fulfill the requirements previously established.

Chapter 4 describes the physical construction of the detection board, software development for all subsystems, and integration challenges. This chapter documents practical engineering solutions and implementation details for replication.
The fifth chapter presents methodology and results from testing the completed system, analyzing detection accuracy, movement precision, and overall system performance against established metrics. Performance limitations and their causes are identified.

Chapter 6 evaluates achievement of research objectives, explores technical challenges encountered, examines practical applications beyond academic context, and acknowledges current system limitations.
The final chapter summarizes the project contributions, proposes directions for future work, and offers final reflections on the development of the chess-playing robot system.
