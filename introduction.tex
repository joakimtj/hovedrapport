
\chapter{Introduction}
\label{chap:intro}

Chess-playing robots represent a fascinating intersection of robotics, computer spacial recognition, and artificial intelligence. These systems combine mechanical precision with computational intelligence to create machines capable of engaging in one of humanity's oldest strategic games. While chess computers have famously defeated grandmasters since the late 1990s, the physical manipulation of chess pieces by robotic systems presents unique engineering challenges beyond pure computational play.

In the introduction, you should do the following, in approximately this order:
\begin{compactitem}
  \item State the subject of your document as clearly as possible, and briefly explain why you are doing it (motivation)
  \item Provide necessary and relevant background information (you will elaborate on this matter in Chapter \ref{chap:analysis}, Analysis)
  \item Define the problem you are addressing, your approach to the problem, and why this problem is important
  \item Define the scope of your work (in particular limitations and things you will {\em not} deal with)
  \item Describe your research method, i.e., how you are going to proceed to answer your research question
  \item Give an outline of the rest of the document
\end{compactitem}

You should think of the introduction  as the foundation of your master work, and accordingly you should put considerable efforts in getting it right.


\section{Background and motivation}
\label{sec:background-motivation}

The development of chess-playing robots has evolved along several technological paths. Previous research has explored various approaches for piece detection and movement systems. For instance, solutions demonstrate a vision-based detection system using overhead cameras to identify piece positions and a robotic arm for movement, while other papers also shed light on systems utilizing magnetic sensors with a sliding mechanism for piece manipulation rather than an articulated robotic arm.
Our project aims to combine the strengths of these approaches while introducing novel elements. By implementing hall effect sensors for piece detection and integrating a robotic arm for movement, we create a system that offers reliable position tracking without the lighting dependencies of vision systems, while maintaining the flexibility of a multi-axis robotic manipulator for piece movement.

TODO: Further elaboration needed on the specific advantages of hall effect sensors over camera-based solutions and why this approach is interesting

\subsection{Research question/Problem statement/Objectives}
\label{sec:research-question}

The primary objective of this project is to develop an integrated chess-playing robot system.
\begin{description}
  \item[Objective 1] To provide a seamless interface between the detection system, chess engine, and robotic control mechanisms
        \begin{description}
          \item[Objective 1.1] To accurately detect the position of all chess pieces on the board using hall effect sensors
          \item[Objective 1.2] To process the board state and determining optimal moves using the Stockfish chess engine
          \item[Objective 1.3] To precisely manipulate chess pieces using the supplied robotic arm
        \end{description}
\end{description}

TODO: Further elaboration needed on specific performance metrics and success criteria

\subsection{Method}
\label{sec:method}

Our approach to developing a chess-playing robot system involves three primary components working in concert:
\begin{itemize}
  \item A chessboard equipped with hall effect sensors to detect piece positions
  \item A computational interface running Stockfish to calculate optimal chess moves
  \item A robotic arm (provided by the school) for physical piece manipulation
\end{itemize}

The development process includes designing the sensor array for the chessboard, creating software interfaces between all components, calibrating the robotic arm for precise movement, and extensive testing of the integrated system.

TODO:  Further elaboration needed on specific technical implementation details and development methodology

\subsection{Deliverables}
\label{sec:deliverables}

The project will result in the following deliverables:
\begin{itemize}
  \item A functional chess-playing robot system integrating all three components
  \item Software for the detection system and computational interface
  \item Technical documentation of the system architecture and integration
  \item Performance analysis and evaluation
  \item This comprehensive report documenting our approach, implementation, and findings
\end{itemize}

\section{Report Outline}
\label{sec:outline}

The last point in the introduction is an outline of the rest of the report, for example like this (from \cite{kjeldsen05cor}):

\begin{quotation}
  Chapter 2 provides background information on range search and line simplification. In the section concerning range search, several data structures and algorithms are presented. The second section describes some of the different techniques developed for performing completely automated line simplification procedures. Finally, another proposed approach to combining the two problems is presented.

  The third chapter gives a general description of the PST and what it can be used for. The interval
  stabbing problem is an important aspect of the work presented in this thesis, and the third chapter
  explains how to solve this with a PST. Next, the interval stabbing problem is expanded to a ``grid
  stabbing problem”, which also can be solved using a PST, and the reason for this is described. Chapter 4 gives a detailed description of the new data structure and the search methods that
  have been developed. After this, theoretical analyses are provided. This chapter also explains how
  an external version of it has been implemented, along with empirical test results to support the theory.

  Chapter 5 presents suggestions for further work. Some work on the suggestions that are made has already been conducted, and this work is also described in this chapter. Finally, there is a chapter providing discussions and conclusions to whether or not the problem can be solved using the approach presented in this thesis.
\end{quotation}

